\section{Introduction}
% No \PARstart
Generating agents with intelligent looking behaviour
has been a constant challenge in the area of Artificial
Intelligence for Video games~\cite{book1}. Among this
conducts, the user expects to appreciate agents than can
perform tactical movements and group strategies. These
tend to be complex in their implementation and usually
result in preestablished moves that can be easily identified
by the user, after several runs of the game. Having agents
execute pathfinding towards a common point, is a problem
that frequently appears in this area. The goal point is
usually given by a location in the game map  (potentially
the opponent's position). A well known and used scheme to
attack this problem is the generation of the minimal path
towards the objective~\cite{art2,book4}. This path is
generated using the game map.  When the algorithm is executed,
it is very likely for the paths to be confluent. Thus,
route diversity and map exploration is avoided.
  
When the minimal path strategy is applied for chasing
the enemy, many escape paths are left open. Therefore,
it is of special interest to generate mechanisms of route
diversification that can produce ambush behaviours.

The techniques explained in this paper are very adaptable
to other various contexts where,  leaving the ambush aside,
it is important to produce path diversity, with the purpose of 
avoiding saturation in certain sectors of the underlying graph. 
Examples of the latter are traffic managers, digital or physical
package routing~\cite{art4}, robotics, among many others.
