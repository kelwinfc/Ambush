\section{A*}

The \astar algorithm \cite{art2,book4,book3} is a variation
of Dijkstra's algorithm \cite{book1} that computes 
paths of minimal cost.
It is an informed search algorithm \cite{book4}, based on 
the following elements:

\begin{itemize}
\item $g$: Represents the accumulated cost from the initial node to the actual node $v$.
\item $\hat{h}$: Is an estimate of the cost from the actual node $v$ to the goal.
\item $\hat{f} = g + \hat{h}$: An estimate of the cost from the initial node to the goal, having $v$ in the path..
\end{itemize}

For the sake of securing optimality, the $\hat{h}$ heuristic 
must be admissible, that is, it shouldn't overestimate any 
cost against the optimal solution.
This algorithm works in a greedy fashion, expanding the next unexplored 
node with the smallest estimated cost $\hat{f}$ at the moment. 
This procedure is repeated until the goal is reached. 

% fin de la nueva def y luego viene lo del orden del algoritmo

The estimated running time of $A^*$ is
$\bigO(|V|log(|V|) + |V|*h + |E|)$.
Having $h$ as the cost of computing function
$\hat{h}$. This cost is derived, assuming an
efficient implementation of the priority queue
such as a Fibonacci heap \cite{book1} and that
the heuristic of each node is only computed
once.