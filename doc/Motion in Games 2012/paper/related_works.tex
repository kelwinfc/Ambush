\section{Related Works}

A similar problem to the one exposed in this paper is
given by \textit{Space-Time A$^*$} \cite{art3}. In this article, 
a variation of \astar is proposed. It works with an undirected
graph, induced from a grid shaped map. The cost function is common
for all agents and constant between every node pair. 

The main goal of the mentioned paper is to avoid having two 
different agents in the same node at a given instant of time.
For this variation of $A^*$, an expansion in the number of dimensions
of \astar	 is proposed: besides the positions of the agents, the elapsed
time is also taken into account. Thus, \textit{Space-Time A$^*$} has a 
considerable increment in the time and memory costs, because the size
of the search space has grown. 

Even though the algorithms have different purposes, \ambush\
shows a clear advantage above (\textit{Space-Time
 A$^*$}) in various different aspects.

In the first place, \ambush\ solves a more general problem 
using a smaller computational complexity. It allows directed 
graphs of variable costs. Also, it can work with agents that
 use different graphs between them.

In addition, if agents move at different speeds,
\textit{Space-Time A$^*$} generates a very dense search
space when the relative speed between elements is high. On the
other hand, this matter doesn't have an effect over the 
performance of \ambush.

If homogeneous costs are assigned to every edge and the increment
on the occupied nodes or edges is made infinite, then the problem that
\ambush\ solves is reduced to the problem proposed for \textit{Space-Time
A$^*$}.
