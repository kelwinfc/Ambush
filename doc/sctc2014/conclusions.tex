\section{Conclusiones y Trabajo Futuro}
\label{sec:conclusions}

\ambush\ y sus variantes muestran una evidente mejora en la conducta de
emboscada, con un c\'omputo adicional poco significativo con respecto
al algoritmo base de generaci\'on de caminos m\'inimos \astar.

Como se pudo apreciar en la secci\'on \ref{sec:experiments}, ninguna
de las variantes de \ambush\ muestra resultados inferiores a los obtenidos
por \astar. Adem\'as, todas las variaciones mostraron no ser gravemente
afectadas por diferencias significativas en el tamaño de los grafos y
del n\'umero de agentes. Dada la diversidad de situaciones planteadas
en los experimentos, se puede afirmar que \ambush\ muestra un comportamiento
estable y efectivo. El tipo de grafos bajo los cuales \ambush\ y sus
variantes presentaron menor eficacia fueron en los grafos con modelo
Dorogovtsev-Mendes, en los cuales, si bien el algoritmo fue capaz de
incrementar significativamente el grado de emboscada, no alcanz\'o
los resultados obtenidos con los otros tipos de grafo.

\ambush\ genera comportamientos de emboscada inteligentes para situaciones
en las que múltiples agentes necesitan alcanzar un objetivo común. En
contraste con estrategias preestablecidas, regularmente implementadas para
situaciones específicas de cada juego que derivan en situaciones repetitivas,
\ambush\ logra fomentar situaciones va\-ria\-das de emboscada dentro del 
comportamiento táctico grupal de los agentes.

En este trabajo se present\'o una nueva m\'etrica para medir el grado de
emboscada que, al utilizar una mayor cantidad de informaci\'on, es capaz
de identificar aquellos casos en los cuales las salidas dejadas al
adversario son de hecho imposible de cubrir por alguno de los agentes.
La m\'etrica propuesta eval\'ua de forma m\'as estable la ca\-li\-dad de
emboscada independientemente de la topolog\'ia del grafo base.
La métrica propuesta en el presente trabajo es mejor que la anterior
para medir el nivel de emboscada, como se refleja en la mayoría de
los experimentos.

En el futuro, se pretende extender este trabajo incluyendo informaci\'on
de capacidad para cada nodo, de tal forma de incluir restricciones
f\'isicas del mundo en el c\'alculo de la ruta. Adem\'as, se pretende
incluir restricciones en la comunicaci\'on que los agentes son capaces
de establecer entre si, para de esta forma poder adaptarse a situaciones
en las cuales existe s\'olo observabilidad parcial entre los agentes.
