\section{Introducci\'on}

En el \'area de Inteligencia Artificial para Videojuegos, la generaci\'on
de conductas inteligentes ha sido un reto constante \cite{MF09}, frecuentemente
derivado en el desarrollo de acciones prestablecidas que el usuario puede
f\'acilmente identificar despu\'es de varias ejecuciones del juego.
Esta caracter\'istica es a\'un m\'as com\'un cuando se trata de la generaci\'on
de movimientos t\'acticos y estrat\'egicos grupales, los cuales suelen ser sumamente
complejos de implementar.

Un problema muy tratado en la literatura es la b\'usqueda de caminos
a un punto com\'un, por parte de grupos de agentes dentro de un juego
\cite{MF09}. Este punto suele venir dado por un lugar en el mapa de juego,
potencialmente la posici\'on del oponente. El esquema regularmente utilizado
es generar caminos de costo m\'inimo \cite{HNR72} \cite{RN93}
hacia este punto, sobre el grafo inducido por el mapa del juego. Es
muy probable, que estos caminos confluyan, evitando la diversidad de
rutas y exploraci\'on del mapa.

Al efectuar una persecuci\'on al oponente, la utilizaci\'on de caminos
\'optimos como estrategia deja muchos espacios de escapatoria libres,
por lo que es de especial inter\'es generar mecanismos de diversificaci\'on de
rutas que generen situaciones de emboscada.

\textbf{
En este punto hablar de nuestra soluci\'on y de como esta prob\'o ser
correcta. Mencionar que este art\'iculo pretende ser un soporte adicional
a los dos anteriores para demostrar la validez de las distintas variantes
de forma exhaustiva as\'i como tambi\'en de presentar una nueva m\'etrica
que permite medir de forma m\'as refinada el grado de emboscada.
}

\begin{comment}
La t\'ecnica expuesta pr\'oximamente es adaptable a muchos contextos en
los que, si bien no es necesaria una situaci\'on de emboscada, es
importante generar diversidad de caminos con el fin de no sobresaturar
ciertos sectores del grafo subyacente. Ejemplo de estos son controladores
de tr\'afico, enrutamiento de paquetes f\'isicos o digitales \cite{TMSV03},
rob\'otica, entre otros.
\end{comment}