\section{Introducci\'on}

La generaci\'on de conductas inteligentes ha sido un reto cons\-tan\-te
en el \'area de Inteligencia Artificial para Videojuegos \cite{MF09},
frecuentemente derivado en el desarrollo de acciones preestablecidas que
el usuario puede f\'acilmente identificar despu\'es de varias ejecuciones
del juego. Esta caracter\'istica se evidencia, m\'as com\'unmente, cuando
se trata de la generaci\'on de movimientos t\'acticos y estrat\'egicos
grupales, los cuales suelen ser sumamente complejos de implementar.

Un problema muy tratado en la literatura es la b\'usqueda de caminos
a un punto com\'un por parte de grupos de agentes dentro de un juego
\cite{MF09}. Este punto suele venir dado por un lugar en el mapa de juego,
potencialmente la posici\'on del oponente. El esquema regularmente utilizado
es generar caminos de costo m\'inimo \cite{HNR72} \cite{RN93}
hacia este punto sobre el grafo inducido por el mapa del juego. Es
muy pro\-ba\-ble que estos caminos confluyan, evitando la diversidad de
rutas y exploraci\'on del mapa.

Al efectuar una persecuci\'on al oponente, la utilizaci\'on de caminos
\'optimos como estrategia deja muchos espacios de escapatoria libres,
por lo que es de especial inter\'es generar mecanismos de diversificaci\'on de
rutas que generen situaciones de emboscada.

Anteriormente, fue propuesta una soluci\'on para este pro\-ble\-ma utilizando
variantes del algoritmo de A* \cite{FGC12e}\cite{FGC12}, las cuales
muestran ser efectivas en el c\'alculo de rutas de emboscada.
El presente trabajo pretende demostrar que dicho enfoque se adapta bien
ante diversos tipos de grafos de distinto tamaño.
Para realizar esta evaluaci\'on, se propone una nueva m\'etrica para validar
la calidad de una emboscada, incorporando m\'as informaci\'on que la m\'etrica
planteada originalmente. Esta nueva m\'etrica incluye informaci\'on de
alcanzabilidad por parte de los agentes, los cuales que pretenden realizar la
emboscada a las salidas dejadas como escapatoria. Esto, con el fin de no
penalizar casos en los cuales el nivel de emboscada alcanzado es el m\'aximo
posible dadas las condiciones iniciales.

El resto de este art\'iculo est\'a organizado como sigue.
En la secci\'on \ref{sec:state_of_the_art} se presentan distintas soluciones
de la literatura para este problema; en la secci\'on \ref{sec:definition}
se muestra la definici\'on formal del problema, en conjunto con la m\'etrica
anterior y la planteada en el presente trabajo para medir la calidad de una
estrategia de emboscada. A continuaci\'on, en la secci\'on \ref{sec:ambush}
son mostrados los algoritmos de \ambush\ y sus variantes, anteriormente
presentados con mayor detalle por Fern\'andez y colaboradores
\cite{FGC12e}\cite{FGC12}. Los ex\-pe\-ri\-men\-tos realizados se incluyen en
la secci\'on \ref{sec:experiments} y, finalmente, las conclusiones y trabajo
futuro son presentados en la secci\'on \ref{sec:conclusions}.
