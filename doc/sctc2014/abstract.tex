\begin{abstract}
En el área de Inteligencia Artificial para videojuegos,
\astar\ es un algoritmo ampliamente utilizado para generar
caminos a ser utilizados por agentes aut\'onomos. En muchos
juegos se espera que los agentes exhiban comportamientos
t\'acticos grupales inteligentes. \ambush\ y sus variantes
resuelven el problema de generaci\'on de emboscadas para
un grupo de agentes basado en \astar. En este trabajo se
presenta una m\'etrica s\'olida para evaluar el resultado
de estos algoritmos con respecto al grado de emboscada.
Adem\'as, se realiza una evaluaci\'on extensa
de la eficacia de estos algoritmos bajo diversas circunstancias.
Se muestra que la m\'etrica planteada presenta mayor solidez
que la m\'etrica original, as\'i como tambi\'en, se valida
la solidez de estos algoritmos.
\\
\\
\textnormal{\textbf{ABSTRACT}}
\\
\\
In the area of Artificial Intelligence for Videogames, \astar\ is a broadly used
algorithm for path-finding by autonomous agents. In several games, it is
expected to observe complex behaviors that involve group tactics. \ambush\ 
and its variations solve the ambush generation problem for a group of
agents using \astar. This work presents a new robust metric to assess
the performance of these algorithms regarding the ambush rate.
Furthermore, the effectiveness of these algorithms is extensively 
proved under different conditions. It is shown that the proposed
metric is more robust than the original one. Also, it is demonstrated
the soundness of \ambush\ and its variations.

%
\begin{keywords}
A*, estrategias grupales, b\'usqueda de caminos, grafos.
\end{keywords}

\end{abstract}