\section{A*, A\*mbush y sus Variantes}
\label{sec:ambush}

En esta secci\'on se muestra el conjunto de adaptaciones realizadas
al algoritmo de \astar, presentadas anteriormente en el art\'iculo
\cite{FGC12e}, para el c\'alculo de emboscadas. Esta secci\'on se
presenta por fined de completitud del trabajo planteado para un
mayor entendimiento de los experimentos realizados.

\subsection{\astar}

El algoritmo de \astar \cite{HNR72}\cite{RN93}\cite{MF09}
es una variación del algoritmo de Dijkstra \cite{CLRS09}
para cómputos de caminos de costo mínimo.
Consiste en un algoritmo de búsqueda informada \cite{RN93},
basado en los siguientes elementos:

\begin{itemize}
\item $g$: Es el costo acumulado desde el nodo inicial a un nodo actual $v$.
\item $\hat{h}$: Estimado del costo desde el nodo actual $v$ a la meta.
\item $\hat{f} = g + \hat{h}$: Estimado del costo desde el nodo inicial a la meta, pasando por $v$.
\end{itemize}

Para garantizar optimalidad, la heuristica $\hat{h}$ debe
ser admisible \cite{HNR72}, es decir, no debe estimar
costos mayores al óptimo del grafo.
El algoritmo actúa de forma voraz, expandiendo el 
si\-guien\-te nodo no explorado con menor costo estimado $\hat{f}$.
Este proceso continúa hasta llegar a la meta. 

En este caso, la complejidad asintótica en tiempo
del algoritmo de A* es de $\bigO(|V|log(|V|) + |V|*h + |E|)$,
con $h$ el costo de cómputo de la función $\hat{h}$,
suponiendo que se cuenta con una implementación eficiente de
cola de prioridades tal como un Fibonacci heap \cite{CLRS09} (estructura
de montículo que permite acceder al mínimo elemento del montículo
y agregar un elemento en tiempo constante; además de
e\-li\-mi\-nar uno en tiempo logarítmico amortizado) y que estamos
computando el costo heurístico de cada nodo una sola vez.

\subsection{\ambush}

A*mbush, presentado como una variaci\'on de A* para el c\'alculo
de emboscadas, consiste en una modificación de la función $g$, que
favorezca la diversidad de caminos, a la cual denominaremos $g'$.
Esta modificaci\'on busca penalizar aquellos nodos/arcos a trav\'es
de los cuales pasen m\'as agentes, para esto, se asume que los
agentes pueden establecer comunicaci\'on total entre ellos para
obtener informaci\'on compartida del c\'alculo de los caminos.

Sea $\Psi(v,i) = 1+(\# j : j \in A \wedge v \in path(j))$,
el número de agentes distintos al agente $i$, que contienen al
nodo $v$ en sus caminos hasta $t$. Se considera que si un agente
no está buscando en dicho momento alcanzar el nodo $t$, o si
aún no ha realizado la búsqueda del camino, este es vacío, por
lo que no se consideran en el cómputo de $\Psi(v,i)$; dado
que $i$ no ha computado ya su camino hasta $t$, se considera
nulo su camino.

Para el nodo inicial, se considera $g'(pos(i),i) = 0$.
Sea $<v,w>$ el siguiente arco a considerar en la expansión del
nodo $v$ en una iteración cualquiera del algoritmo, se considera
$g'(w, i) = g'(v,i) + \lambda_i(<v,w>) \cdot \Psi(w,i)^2$\\

Dado que $\Psi(v,i) \geq 1$, el camino obtenido por \textit{A*mbush}
es óptimo sobre la nueva definición de $g'$, por lo que las
propiedades de $A^*$ se mantienen \cite{HNR72}. Sin embargo, sobre
la función original de costos, el camino obtenido no es necesariamente
óptimo.

Dado que es posible precomputar la funci\'on de incremento de
costos $\Psi$. Si almacenamos los resultados de dicha funci\'on
en una estructura de acceso constante, el costo de computar $g'$
es asintóticamente igual al de $g$, por lo tanto, la única
variación en el costo del algoritmo, viene dada por el
cómputo inicial de la función $\Psi$. Ambas variaciones, tienen
complejidad en tiempo
$\bigO(|V|log(|V|) + |V|*h + |E| + |A|*|V| )$.

En el campo específico de los videjuegos, los grafos
de interés, vienen dados por la división en polígonos
del mapa
\cite{MF09} \cite{CS11}
(regularmente con pocos lados), según las
regiones transitables de éste o cuadrículas y sus
adyacencias \cite{MF09} \cite{CS11}.
Dado que los polígonos tienen un número
reducido de lados, estos grafos, suelen ser poco densos;
es decir, $|E| \in \bigO(|V|)$, por lo que el tiempo
de ejecución de este método, sobre los grafos de interés
en el área de videojuegos, viene dado por
$\bigO(|V|(log(|V|) + h + |A|) )$.

\begin{comment}
Adicionalmente, el número de agentes suele no ser
lo suficientemente grande para afectar significativamente
la complejidad del algoritmo. Casos extremos en los que
sea necesaria la utilización de un gran número de agentes,
se pueden solucionar con la agrupación de éstos bajo líderes
ficticios \cite{MF09}, de modo que por cada grupo se realice un solo
cálculo del camino. Naturalmente, mientras se agrupen los
individuos en conjuntos más grandes, la diversidad de
caminos se verá afectada negativamente.
\end{comment}

\subsection{P-A*mbush}

\subsection{R-A*mbush}

\subsection{SAR-A*mbush}
