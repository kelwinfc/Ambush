\section{A*, A*mbush y sus Variantes}
\label{sec:ambush}

En esta secci\'on se muestra el conjunto de adaptaciones realizadas
al algoritmo de \astar, presentadas anteriormente en el art\'iculo
\cite{FGC12e}, para el c\'alculo de emboscadas. Esta secci\'on se
presenta por fines de completitud del trabajo planteado para un
mayor entendimiento de los experimentos realizados.

\subsection{\astar}

El algoritmo de \astar\ \cite{HNR72}\cite{RN93}\cite{MF09}
es una variación del algoritmo de Dijkstra \cite{CLRS09}
para cómputos de caminos de costo mínimo.
Consiste en un algoritmo de búsqueda informada \cite{RN93},
basado en los siguientes e\-le\-men\-tos:

\begin{itemize}
\item $g$: Es el costo acumulado desde el nodo inicial a un nodo actual $v$.
\item $\hat{h}$: Estimado del costo desde el nodo actual $v$ a la meta.
\item $\hat{f} = g + \hat{h}$: Estimado del costo desde el nodo inicial a la meta, pasando por $v$.
\end{itemize}

Para garantizar optimalidad, la heuristica $\hat{h}$ debe
ser admisible \cite{HNR72}, es decir, no debe estimar
costos mayores al óptimo del grafo.
El algoritmo actúa de forma voraz, expandiendo el 
si\-guien\-te nodo no explorado con menor costo estimado $\hat{f}$.
Este proceso continúa hasta llegar a la meta. 

En este caso, la complejidad asintótica en tiempo
del algoritmo de \astar\ es de $\bigO(|V|log(|V|) + |V|*h + |E|)$,
con $h$ el costo de cómputo de la función $\hat{h}$,
suponiendo que se cuenta con una implementación eficiente de
cola de prioridades tal como un Fibonacci heap \cite{CLRS09} (estructura
de montículo que permite acceder al mínimo elemento del montículo
y agregar un elemento en tiempo constante; además de
e\-li\-mi\-nar uno en tiempo logarítmico amortizado) y que estamos
computando el costo heurístico de cada nodo una sola vez.

\subsection{\ambush}

\ambush, presentado como una variaci\'on de \astar\ para el c\'alculo
de emboscadas\cite{FGC12e}, consiste en una modificación de la
función $g$, que favorezca la diversidad de caminos, denominada $g'$.
Esta modificaci\'on busca penalizar aquellos nodos/arcos a trav\'es
de los cuales pasen m\'as agentes, a\-su\-mien\-do que estos pueden
establecer comunicaci\'on total entre ellos para
obtener informaci\'on compartida del c\'alculo de los caminos.

Sea $\Psi(v,i) = 1+(\# j : j \in A \wedge v \in path(j))$,
el número de agentes distintos al agente $i$, que contienen al
nodo $v$ en sus caminos hasta $t$. Se considera como nulo los
caminos a\'un no computados de los otros agentes.
Para el nodo inicial, se considera $g'(pos(i),i) = 0$.
Sea $<v,w>$ el si\-guien\-te arco a considerar en la expansión del
nodo $v$ en una iteración cualquiera del algoritmo, se considera
$g'(w, i) = g'(v,i) + \lambda_i(<v,w>) \cdot \Psi(w,i)^2$.

Dado que $\Psi(v,i) \geq 1$, el camino obtenido por \ambush\
es óptimo sobre la nueva definición de $g'$, por lo que las
propiedades de $A^*$ se mantienen \cite{HNR72}. Sin embargo, sobre
la función original de costos, el camino obtenido no es necesariamente
óptimo. En t\'erminos de eficiencia del algoritmo, si se precomputa
la funci\'on de incremento de costos $\Psi$, el costo de computar
$g'$ es asintóticamente igual al de $g$, por lo tanto, la única variación
en el costo del algoritmo con respecto a \astar, viene dada por el cómputo
inicial de la función $\Psi$, dando como complejidad en tiempo
$\bigO(|V|log(|V|) + |V|*h + |E| + |A|*|V| )$.

En el campo específico de los videjuegos, los grafos
de interés vienen dados por la división en polígonos
del mapa (regularmente con pocos lados), según las
regiones transitables de éste o cuadrículas y sus
adyacencias \cite{MF09} \cite{CS11}.
Dado que los polígonos tienen un número
reducido de lados, estos grafos suelen ser poco densos;
es decir, $|E| \in \bigO(|V|)$, por lo que el tiempo
de ejecución de este método, sobre los grafos de interés
en el área de videojuegos, viene dado por
$\bigO(|V|(log(|V|) + h + |A|) )$.

\subsection{\pambush}

Los caminos calculados por \ambush\ pueden verse m\'as o menos
naturales dependiendo del orden en el cual los agentes computan
las rutas. Por ejemplo, si un agente cercano al oponente escoge
su camino de \'ultimo, este pro\-ce\-de\-r\'a a recorrer una ruta
larga dando prioridad a la diversificaci\'on, mientras que
agentes alejados, por los cuales el jugador no suele tener un
foco de atenci\'on alto, toman rutas \'optimas hacia el objetivo.
\pambush\ se trata de una variaci\'on de \ambush\ que considera el
orden del c\'omputo de los caminos de acuerdo a un m\'etodo de
asignaci\'on de prioridades P. Este m\'etodo puede considerar
no s\'olo la distancia, sino tambi\'en otros factores dependientes
de la aplicaci\'on particular. En este trabajo se considera la
prioridad de acuerdo a la distancia al destino.

Otra ventaja, al considerar este m\'etodo de prioridades, es que
\pambush\ alcanza siempre la meta en el menor tiempo posible por al
menos uno de los agentes. A pesar de que el costo de calcular
\pambush\ con distancia real es mayor que el costo de \ambush\ 
para un \'unico agente, el costo amortizado del c\'alculo de
todos los agentes es el mismo que el de \ambush. Se sugiere
utilizar otro tipo de distancias como la euclidiana en caso
de ser necesario disminuir los tiempos de respuesta.

\subsection{\rambush}

Buscando minimizar el incremento en el costo de los caminos,
se present\'o \rambush, el cual plantea ejecutar \ambush\ una
vez alcanzada una cierta distancia m\'inima al objetivo ($R$) y,
antes de alcanzar este punto, utilizar \astar. De esta forma,
se evita que los agentes escojan rutas in\-ne\-ce\-sa\-ria\-men\-te
largas desde un inicio.

Sin embargo, esta soluci\'on es menos general que la o\-ri\-gi\-nal,
dado que requiere la incorporaci\'on del radio $R$, el cual es
variable para cada posible mapa de juego y disposici\'on de los
agentes. Es por esto que se plantea la variante \sarambush, la
cual incorpora la selecci\'on del mejor radio $R$ a partir del
cual los agentes comienzan a ejecutar \ambush.

\subsection{\sarambush}

\sarambush\ es la \'ultima variaci\'on estudiada, la cual
resuelve el problema de selecci\'on del radio $R$ para el
algoritmo de \rambush. Su nombre proviene de Self Adaptive
\rambush. Este m\'etodo hace calcular inicialmente a cada
agente su camino hasta el destino utilizando \astar. Luego,
un conjunto de puntos es escogido de entre los nodos recorridos
en el camino m\'inimo como posibles candidatos para establecer
el valor de $R$. El algoritmo escoge como mejor $R$ aquel
con m\'inimo radio que incremente el grado de emboscada. En
caso de empates, el algoritmo considera el balance de los
agentes (\ref{eq:distr}) a trav\'es de los predecesores del
nodo objetivo ($pred(t)$), con el fin de lograr una distribuci\'on m\'as
equilibrada de \'estos. 
En la f\'ormula presentada (\ref{eq:distr}), $n$ representa
el n\'umero de agentes que ya han calculado el camino y
$num(i)$ es el n\'umero de agentes que alcanzan a $t$ a trav\'es
de $i$.
La escogencia del menor radio se
centra en empezar el camino sub\'optimo tan tarde como sea
posible. Es importante notar que \rambush\ se comporta exactamente
igual \astar\ cuando $R=0$ y como \ambush\ cuando $R$ es mayor
o igual que la distancia a la meta del agente.

\begin{small}
\begin{equation}
1-\dfrac{\left(\Sigma i |  i \in pred(t)
  						\wedge num(i) > 
  							\left\lceil \frac{n}{|pred(t)|} \right\rceil 
  					:  num(i) - 
  						\left\lceil \frac{n}{|pred(t)|} \right\rceil\right)}
  						{n-\left\lfloor \frac{n}{|pred(t)|} \right\rfloor}
\label{eq:distr}
\end{equation}
\end{small}

La eficiencia de esta variante esta condicionada al n\'umero de
nodos considerados como candidatos a radio. Con el fin de demostrar
el potencial m\'aximo del algoritmo, se con\-si\-de\-ra cada posible nodo
como candidato. Sin embargo, para implementaci\'on con casos reales, 
se aconseja utilizar un n\'umero fijo constante o utilizar alg\'un
criterio como selecci\'on de los nodos que se encuentran en las
posiciones que son potencia de dos; de esta forma, se busca minimizar
el impacto de recomputar el camino para cada uno.
