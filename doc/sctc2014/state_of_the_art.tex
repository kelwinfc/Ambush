\section{Trabajos Relacionados}
\label{sec:state_of_the_art}

Este trabajo se centra en el estudio del m\'etodo expuesto
previamente por Fern\'andez y colaboradores \cite{FGC12e}\cite{FGC12}.
Enfoques alternativos a \'este fueron propuestos por Silver \cite{Sil06}
y por van Toll y colaboradores \cite{TCG12}.

En el trabajo propuesto por Silver \cite{Sil06}, denominado
\textit{Space-Time \astar}, se plantea una variante de $A^*$ 
donde se tiene un grafo no dirigido en forma de cuadrícula,
con función de costos común para todos los agentes y 
constante entre cada par de nodos.
El objetivo principal de dicho trabajo es evitar el paso de
dos agentes por un mismo nodo en un instante de tiempo dado.
La variación planteada incluye una extensión en el número
de dimensiones de \astar. Se considera, además de la posición
de los agentes, el tiempo transcurrido. Dicha variación tiene
un costo añadido en tiempo y memoria considerable debido a que
se incrementa el tamaño del espacio de búsqueda. Aunque el objetivo
final de las soluci\'on es diferente, la estudiada en el actual
trabajo presenta una clara ventaja sobre \textit{Space-Time A$^*$}
en varios aspectos. En primer lugar, resuelve con un algoritmo
de menor complejidad de c\'omputo problemas m\'as generales, debido a
que permite la utilizaci\'on de grafos dirigidos con costos de arcos
variables. Asimismo, se puede trabajar con agentes que
tengan grafos distintos entre sí.
Por otro lado, en el caso de tener agentes con distintas velocidades,
\textit{Space-Time A$^*$} genera un espacio de búsqueda en extremo
denso, cuando la velocidad relativa entre los e\-le\-men\-tos del juego es
alta. En cambio, $A^*mbush$ no se ve afectado por la diferencia de 
velocidades entre los agentes.
Asignando costos homogéneos para cada arco e incremento infinito
sobre nodos o arcos ocupados, el problema que resuelve $A^*mbush$
se reduce al problema planteado para \textit{Space-Time
A$^*$}.

El segundo trabajo mencionado, desarrollado por van Toll y otros
\cite{TCG12}, muestra un enfoque basado en densidades para generar
diversidad de caminos. Sin embargo, la diversificaci\'on de caminos
es dada s\'olo al alcanzar un gran n\'umero de agentes atravesando una
misma \'area, por lo que se torna infactible en los casos estudiados
en este trabajo, donde se pretende alcanzar emboscadas utilizando
el m\'inimo n\'umero de agentes posible.
